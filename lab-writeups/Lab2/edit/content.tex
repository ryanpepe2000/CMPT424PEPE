%-----------------------------------------------------------------------------------------------------
%TABLE OF CONTENTS (COMMENT OUT THIS SECTION IF YOU DON'T WANT TO ADD TABLE OF CONTENTS)
%-----------------------------------------------------------------------------------------------------
\tableofcontents 
\newpage
%-----------------------------------------------------------------------------------------------------

%-----------------------------------------------------------------------------------------------------
%REPORT TEXT
%-----------------------------------------------------------------------------------------------------
%\section{Introduction}
%    \lipsum[1,1]
\section{Questions}
    \begin{enumerate}
        \item How is your console like the ancient TTY subsystem in Unix as described in www.linusakesson.net?
        \item[] The console in THaNatOS is quite similar to the TTY subsystem in Unix as described by the provided article. For instance, the subsystem in Unix uses a keyboard interrupt process that writes to a buffer. This buffer is written to the screen on a CPU clock cycle and is deleted in a similar fashion to the backspace that I implemented. Additionally, these interrupts are essentially a mechanism for the user to communicate asynchronously with a process through the kernel. The buffer in the THaNatOS system acts as the "pseudo terminal" described in the article. In the Unix subsystem, there is a "frame buffer" that writes to a VGA display; however, in my system, the system buffer is written to an HTML canvas. 
    \end{enumerate}
%\section{Conclusion}
%    \lipsum[4-4] %dummy text, replace with original text
%-----------------------------------------------------------------------------------------------------

%-----------------------------------------------------------------------------------------------------
%BIBLIOGRAPHY (COMMENT OUT THIS SECTION IF YOU DON'T WANT TO ADD BIBLIOGRAPHY)
%-----------------------------------------------------------------------------------------------------
%
%\begin{thebibliography}{}%
%
%\bibitem{reference_label1}
%{2.Author}. \textit{1. Name of the document}. {Magazine/Journal/Publisher}.\\
%\texttt{link to the web resource}

%\bibitem{reference_label2}
%{2.Author}. \textit{2. Name of the document}. {Magazine/Journal/Publisher}.\\
%\texttt{link to the web resource}

%\end{thebibliography}
%-----------------------------------------------------------------------------------------------------